
\section{Introduction} %for journal use above \firstsection{..} instead

\subsection{Motivation}

\subsubsection{Why Health Interventions?}
% Why should society care about health interventions?
Despite significant advances in the theory and practice of behavioral science, humans continue make poor behavioral choices on a daily basis, and the reasons for those choices remain, to some extent, uncharted. 
The consequences of these daily choices are often insignificant in the moment but over time build up to larger individual and societal problems. 
In fact, many of today's greatest health challenges can be mapped back to individual behavioral choices.
A habitual lack of physical activity, poor diet, and smoking is likely to lead to a variety of health problems (e.g., obesity, diabetes, heart disease, cancer, chronic pain, depression), lower quality of life and shortened lifespans \cite{franco2005, dunn2001, yanbaeva2007, ross2000}. 
Smoking and lack of exercise are behaviors that clearly affect health and longevity (by increasing the chance of getting cancer, or decreasing the chance of obesity and diabetes), but the general populus seems to lack power over their own habits.
Similar challenges exist outside the realm of personal health. 
Academic success is a function of attending class and completing assigned tasks, among other daily behaviors\cite{cooper2006}.
In personal finance, poor day-to-day purchasing decisions can add up to large financial debts\cite{norvilitis2003}.

Behavior change is the art and science of facilitating adoption of new behaviors or abandoning of old behaviors.
Behavior change interventions are particularly important to modern health care systems in developed countries, where preventative behavioral treatment receives little attention from heath care practitioners.


\subsubsection{Why Do Intervention Dynamics Matter?}
The confluence of pervasive sensing, machine learning, network access, and computation facilitates new approaches to guiding behavioral choices. 
These systems can detect behaviors and psychological states such as stress\cite{chang2011,lu2012}, physical activity\cite{li2010,emken2012}, social interaction\cite{wyatt2011}, and smoking\cite{sazonov2011}, automatically and, in some cases, in real-time.
These data provide new opportunities for the human-computer interaction (HCI), behavioral science, and other related communities to develop user interfaces for mobile behavioral interventions that help users make better in the moment behavioral choices related to health\cite{klasnja2012,nahum2012}, productivity\cite{ho2005,sohn2005,jewell2011}, personal finance\cite{gallego2012}, and environmental stewardship.\cite{elliott2012}
New methods for evaluating these behavioral interventions remain underexplored and conventional methods of analysis do not offer the level of detail needed to explore the implicit dynamics of just-in-time, interactive, or adaptive interventions.

\subsubsection{Contributions}
% summarize previous work (1 paragraph)
In addition to metrics of success of an intervention, behavioral theorists need tools to help understand the dynamics of behavioral responses to a stimulus.
Due to the lack of a dynamical treatment of behavior within theories, existing models behavior appear inadequate to inform state-of-the-art intervention development \cite{riley2011}.
Applicable methods of intervention analysis and data visualization have been slow to reach behavioral researchers, dramatically limiting their ability to develop of state-of-the-art behavioral theories to address these shortcomings.

% summarize work done (1 paragraph - maybe a long one)
In order to help bridge the gap between state-of-the-art data analysis and behavioral research, in this paper we present methods for exploring the dynamical responses of behavior-change intervention events built on a foundation of user-driven design principles.
Through a survey of 11 behavioral scientists and countless consultations with domain experts on the cutting edge of behavioral intervention design, we have identified and attempted to address a set of goals and key tasks which are of special interest to researchers in this area. 

% summarize contributions (1 paragraph - maybe a long one)
In this paper we present:
% Bulleted list of 2-3 contributions
\begin{enumerate}
	\item{a set of design considerations to guide visualization and tool design in this domain}
	\item{introduction to intervention response dynamics in relation to developing theories}
	\item{visualization methods which address some key tasks addressing the goals identified in the design considerations section}
\end{enumerate}

% an alternative framing of of our contributions: 
\comment{
	- to understand how we can use this type of data to accomplish these domain-specific tasks?
	- to propose new hypotheses and techniques to support these tasks?
	- to propose lessons learned that may apply for similar viz problems.
}

% highlight why should viz community care!
This work breaks new ground in the application domain, but also provides guidelines for future work bridging the data visualization and behavioral science communities.

