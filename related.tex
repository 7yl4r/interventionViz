\section{Previous Work}
Much work exists on both behavioral intervention analysis and event-based time series visualization.
In this work we draw inspiration from the cutting edge of both disciplines and explore the application of existing data visualization methods on the novel domain of behavioral intervention dynamics.

\subsection{Related Event-based Time Series Visualization}
"Lifelines" \cite{plaisant1996} allow for the exploration of events in a series for one individual, and new research in event sequence analysis \cite{Wongsuphasawat2011}, including analysis of event patterns \cite{Wongsuphasawat2012, fails2006, vrotsou2007} and the relation of multiple symptoms \cite{wongsuphasawat2011outflow} helps researchers examine outcomes on a "macro-scale" across many subjects by aggregating records into a single view. 
Additionally, the problem of identifying patterns at multiple time scales has been partially addressed through clustering of time series \cite{van1999}, and methods for exploring the "paths" traversed by many individuals between many event types and statistical analyses to highlight relationships between events has recently been established \cite{gotz2014}.

To our knowledge, little existing work addresses the dynamics of a numerical variable's response to a behavioral intervention event. 
Statistical methods of intervention analysis \cite{box1975}, have been applied across various disciplines but thus far there has been little demand for these methods in behavioral science.
Only recently has new wearable sensing technology made time-intensive, in-the-wild measurements feasible.
Additionally, the prospect of ubiquitous intervention delivery via mobile devices and the concept of Just-in-Time-Adaptive-Interventions (JiTAIs) have introduced a new demand for a more detailed understanding of human behavior.

\subsection{Current Intervention Methods}
Recent advances in sensing and ubiquitous computing are enabling examination of and influence over behavior at small time scales (on the order of seconds) and in a wide range of daily life contexts.
New wearable sensing technologies are changing the way we do experiments, and mobile phones are a powerful new medium for delivering behavioral interventions "just-in-time".

Recent works have explored "adaptive interventions" tailored based on "tailoring variables" which may include user preferences, context, and personality \cite{collins2004}. 
Despite these drastic changes in the interventions, methods for evaluating adaptive interventions remain in many ways similar to "fixed" interventions \cite{collins2004}, and the use of the multiphase optimization strategy (MOST) and sequential multiple assignment randomized trials (SMART) \cite{collins2007} maximize efficiency in applying these methods.
These methods become more difficult to apply, however, when dealing with Just in Time Adaptive Interventions (JiTAIs).

Existing work on visual analysis of systems usability \cite{harrison1994} may be applied to the evaluation of JiTAI systems, however, these methods focus largely on a single record, rather than generalizations drawn from across many.
Additionally, little theoretical guidance in terms of effect latency or delay exists to aid in the planning or analysis of experimental trials.

A theoretical basis which takes dynamical effects into consideration to enable improved behavioral intervention optimization has been proposed for interventions mediating gestational weight gain \cite{dong2013}, smoking behavior \cite{timms2014}, childhood anxiety \cite{pina2014}, and fibromyalgia \cite{deshpande2014}.
This new type of theoretical model is most effective on the timescale of multiple days, weeks, or months - partially because the confounds of contextual, within-day variations make analysis at this level difficult, but mostly because theories of behavior at this time scale are underdeveloped.
Methods for analyzing the dynamics of intervention responses using existing data are needed in order to catalyze the development of theories to explain these signals. 
